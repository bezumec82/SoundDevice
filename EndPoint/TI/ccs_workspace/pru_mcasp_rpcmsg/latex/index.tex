This is the project to read-\/write Mc\-A\-S\-P from P\-R\-U on T\-I Sitara devices. P\-R\-U directly accesses register of dedicated Mc\-A\-S\-P devices. 